%%%%%%%%%%%%%%%%%%%%%%%%%%%%%%%%%%%%%%%%%%%%%%%%%%%%%%%%%%%%%%%%%%%%%%%%
%                              Preamble                                %
%%%%%%%%%%%%%%%%%%%%%%%%%%%%%%%%%%%%%%%%%%%%%%%%%%%%%%%%%%%%%%%%%%%%%%%%

% ----------------------------------------------------------------------
% Set the document class
% ----------------------------------------------------------------------
\documentclass[12pt]{article}

% ----------------------------------------------------------------------
% Define external packages, language, margins, fonts, new commands 
% and colors
% ----------------------------------------------------------------------
\usepackage[utf8]{inputenc} % Codification
\usepackage[english]{babel} % Writing idiom

\usepackage[export]{adjustbox} % Align images
\usepackage{amsmath} % Extra commands for math mode
\usepackage{amssymb} % Mathematical symbols
\usepackage{anysize} % Personalize margins
    \marginsize{2cm}{2cm}{2cm}{2cm} % {left}{right}{above}{below}
\usepackage{appendix} % Appendices
\usepackage{cancel} % Expression cancellation
\usepackage{caption} % Figure numeration
\usepackage{cite} % Citations, like [1 - 3]
\usepackage{color} % Text coloring
\usepackage{fancyhdr} % Head note and footnote
    \pagestyle{fancy}
    \fancyhf{}
    \fancyhead[L]{\footnotesize IST} % Left of Head note
    \fancyhead[R]{\footnotesize ULisboa} % Right of Head note
    \fancyfoot[L]{\footnotesize SIRS} % Left of Footnote
    \fancyfoot[C]{\thepage} % Center of Footnote
    \fancyfoot[R]{\footnotesize METI} % Right of Footnote
    \renewcommand{\footrulewidth}{0.4pt} % Footnote rule
\usepackage{float} % Utilization of [H] in figures
\usepackage{graphicx} % Figures in LaTeX
\usepackage[colorlinks = true, plainpages = true, linkcolor = istblue, urlcolor = istblue, citecolor = istblue, anchorcolor = istblue]{hyperref}
%\usepackage{indentfirst} % First paragraph
\usepackage{parskip}
\usepackage{siunitx} % SI units
\usepackage{subfigure} % Subfigures
\setlength {\marginparwidth }{2cm} 
\usepackage{todonotes}
\usepackage[font=footnotesize,labelfont=bf]{caption}

% Random text (not needed)
\usepackage{lipsum}
\usepackage{duckuments}
\usepackage{hyperref}

% New and re-newcommands
\newcommand{\sen}{\operatorname{\sen}} % Sine function definition
\newcommand{\HRule}{\rule{\linewidth}{0.5mm}} % Specific rule definition
\renewcommand{\appendixpagename}{\LARGE Appendices}

% Colors
\definecolor{istblue}{RGB}{3, 171, 230}
\definecolor{dkgreen}{rgb}{0,0.6,0}
\definecolor{gray}{rgb}{0.5,0.5,0.5}

%%%%%%%%%%%%%%%%%%%%%%%%%%%%%%%%%%%%%%%%%%%%%%%%%%%%%%%%%%%%%%%%%%%%%%%%
%                                 Document                             %
%%%%%%%%%%%%%%%%%%%%%%%%%%%%%%%%%%%%%%%%%%%%%%%%%%%%%%%%%%%%%%%%%%%%%%%%
\begin{document}

% ----------------------------------------------------------------------
%                               CAPA
% ----------------------------------------------------------------------
\begin{titlepage}
    \centering
    
    % Logo do IST (alinhado à esquerda como costuma ser no template oficial)
    \begin{flushleft}
        \includegraphics[width=5cm]{images/istlog.png}
    \end{flushleft}

    \vspace{2.0cm}

    % Título Principal: Usa Negrito em vez de Small Caps para maior impacto
    {\huge \textbf{Design and Implementation of an Authenticated Post-Quantum Key Exchange Protocol} \par}
    
    \vspace{1.0cm}
    
    % Subtítulo: Usar Small Caps apenas aqui ou Itálico para criar contraste
    {\Large \textit{An Adaptation of the Encrypted Key Exchange Protocol using CRYSTALS-Kyber} \par}

    \vspace{2.5cm}

    % Linhas e Grau: Mais finas e elegantes
    \rule{\linewidth}{0.2pt} \\[0.4cm]
    {\large Master's degree in Telecommunications and Informatics Engineering} \\[0.4cm]
    \rule{\linewidth}{0.2pt}

    \vfill

    % Informação de baixo
    {\large \textbf{Author:}} \\
    {\large Inês Martins Ribeiro} \\
    
    \vspace{0.8cm}
    
    {\large \textbf{Professor}} \\
    {\large Prof. Ricardo Chaves}

    \vspace{1.5cm}

  %  {\large \textbf{Draft}} \\ % Ou a data final
   % {\large December 2025}

\end{titlepage}


\thispagestyle{empty}

\setcounter{page}{0}

\newpage

% ----------------------------------------------------------------------
% Indíce
% ----------------------------------------------------------------------
\renewcommand*\contentsname{Indíce}
\tableofcontents


\newpage

% ----------------------------------------------------------------------
% Body
% ----------------------------------------------------------------------
\section{Introduction}

% ----------------------------------------------------------------------
% Backgroung
% ----------------------------------------------------------------------


\end{document}
